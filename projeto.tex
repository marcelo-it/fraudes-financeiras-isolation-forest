\documentclass[12pt,a4paper]{article}
\usepackage[utf8]{inputenc}
\usepackage[brazil]{babel}
\usepackage[T1]{fontenc}    
\usepackage{graphicx}
\usepackage{hyperref}
\usepackage{abnt-alf}
\usepackage[top=3cm,bottom=2cm,left=3cm,right=2cm]{geometry}
\usepackage{indentfirst}

\begin{document}

% CAPA
\pagestyle{empty}
\begin{center}
\large  \textbf{UNIVERSIDADE PRESBITERIANA MACKENZIE} \\
\large  \textbf{PROGRAMA DE PÓS-GRADUAÇÃO EM}\\
\large  \textbf{ENGENHARIA ELÉTRICA E COMPUTAÇÃO}\\
\vskip 2.0cm
\textbf{\large João Marcelo Cattaldo Amorim}\\
\vskip 4.0cm
\setlength{\baselineskip}{1.5\baselineskip}
\textbf{\large IDENTIFICAÇÃO DE GOLPES FINANCEIROS EM INDÚSTRIAS DE BENS DE CONSUMO NA CONCESSÃO DE CRÉDITO A PESSOAS  JURÍDICAS COM USO DE INTELIGÊNCIA ARTIFICIAL}\\
\vskip 4.5cm
\end{center}
\hfill{\vbox{\hsize=8.5cm\noindent\strut
Projeto de Pesquisa apresentado ao Programa\break
de Pós-Graduação em Engenharia Elétrica e\break
Computação da Universidade Presbiteriana\break
Mackenzie como parte dos requisitos para a\break
aprovação na disciplina de Metodologia do\break
Trabalho Científico.}\\
\strut}
\vskip 3.0cm
\textbf{\normalsize Orientador: Prof. Dr. Leandro Augusto da Silva}\\
\vskip 2.0cm
\begin{center}
São Paulo\\
2024\\
\end{center}

% RESUMO
\newpage
\thispagestyle{plain}
\pagenumbering{roman}
\begin{center}
\large
\textbf{RESUMO}
\end{center}
\renewcommand{\baselinestretch}{0.6666666}
Esta dissertação investiga a identificação de padrões associados a golpes financeiros em indústrias de bens de consumo que concedem crédito a clientes pessoas jurídicas para a aquisição de seus produtos, utilizando técnicas de inteligência artificial, com implementação no fluxo operacional do motor de crédito.
A análise da base de dados revelou que aproximadamente 15\% dos clientes — cerca de 148.000 empresas — encontram-se em estado de inadimplência não recuperada, representando mais de R\$ 2,1 bilhões em dívidas. Esses clientes possuem um score de crédito classificado como de alto risco.
O modelo de crédito atualmente utilizado pela empresa apresenta limitações significativas, pois desconsidera fatores críticos que podem indicar golpes na concessão de crédito. Enquanto o rating score tradicional baseia-se principalmente no histórico de pagamentos e nas restrições associadas ao cliente, golpes financeiros frequentemente envolvem alterações cadastrais recentes, como mudanças de endereço ou código de atividade econômica, além de comportamentos atípicos, como aumentos súbitos no volume de compras e consultas, e discrepâncias entre o tempo de atividade declarado e a data de fundação da empresa.
Para enfrentar esses desafios, esta pesquisa desenvolveu uma solução baseada em modelos de \textit{machine learning}, utilizando técnicas avançadas de detecção de anomalias, como o Isolation Forest. Essa técnica destacou-se por validar as principais hipóteses relacionadas aos padrões de fraudes. A abordagem demonstrou eficiência na identificação de comportamentos atípicos e padrões indicativos de golpes financeiros, contribuindo significativamente para o aprimoramento da avaliação de risco de crédito e para a mitigação de perdas financeiras associadas a fraudes.
\\[0.5cm]
\begin{flushleft}
{\bf Palavras-chave:} {Análise de Crédito, Risco de Crédito, Aprendizado de Máquina, Isolation Forest, Detecção de Fraude, Golpe Financeiro e CRISP-DM.}
\end{flushleft}

% SUMÁRIO
\newpage
\thispagestyle{empty}
\tableofcontents

% INTRODUÇÃO
\newpage
\pagestyle{plain}
\pagenumbering{arabic}
\renewcommand{\baselinestretch}{1.5}
\normalsize
\section{INTRODUÇÃO}
A CISP (Central de Informações São Paulo) é uma associação sem fins lucrativos fundada em 1972, com a missão de oferecer soluções exclusivas para análise de risco de crédito, contribuindo para o desenvolvimento econômico nacional. Composta por 192 grandes indústrias de produtos de largo consumo, organizadas em oito segmentos (Alimentos, Bebidas, Higiene Pessoal e Cosméticos, Papel, Papelaria, Utilidades Domésticas, Eletroeletrônicos e Produtos de Limpeza), a CISP representa aproximadamente 8\% do PIB brasileiro, destacando-se como uma entidade de grande relevância no setor.

O sistema da CISP opera exclusivamente com clientes pessoas jurídicas, abrangendo cerca de 1.400.000 CNPJs. Os associados fornecem periodicamente informações comerciais de seus clientes, enriquecidas com dados de fontes públicas, como Receita Federal, Sintegra, Suframa e Protestos. Com base nessas informações, é gerada uma classificação de risco de performance (Rating Score), que atribui notas de A (menor risco) a E (maior risco). Atualmente, 1.700 usuários das áreas de crédito e cobrança utilizam os relatórios da CISP, acessados manualmente, por API ou de forma automatizada via plataforma Maxxi.

No entanto, fóruns realizados pela CISP com profissionais das áreas de crédito e cobrança identificaram limitações no modelo atual, que não considera fatores importantes relacionados a golpes financeiros e inadimplência. Exemplos incluem crescimento atípico de compras, alterações cadastrais recentes e discrepâncias entre a data de fundação e o tempo de atividade. Dados preliminares revelam que 15\% dos clientes — cerca de 148.000 empresas — encontram-se em inadimplência não recuperada, acumulando R\$ 2,1 bilhões em dívidas.

O objetivo deste estudo é propor e validar um modelo de inteligência artificial para identificar fraudes financeiras, implementar a solução no fluxo atual do motor de crédito e mitigar os riscos enfrentados pelos associados da CISP. O modelo complementará o sistema atual, incorporando fatores específicos relacionados a golpes financeiros e inadimplência elevada. Além de reduzir perdas financeiras, o estudo busca contribuir para o avanço das práticas de gestão de risco de crédito no setor e para a literatura acadêmica.

Para garantir a confidencialidade e a conformidade legal, o estudo utilizou dados anonimizados por meio do algoritmo de hash SHA-256, que transforma os CNPJs dos clientes em valores únicos e irreversíveis. A preparação dos dados e o desenvolvimento do modelo seguiram o framework CRISP-DM (\textit{Cross-Industry Standard Process for Data Mining}), uma metodologia amplamente reconhecida para projetos de ciência de dados.

Esta pesquisa utiliza algoritmos de inteligência artificial para detectar fraudes financeiras e padrões de inadimplência, oferecendo uma solução técnica robusta que contribui tanto para a segurança financeira quanto para a inovação na gestão de risco de crédito.

% REFERENCIAL TEÓRICO
\newpage
\section{REFERENCIAL TEÓRICO}
\label{sec:referencial}

O referencial teórico apresenta as bases conceituais e metodológicas que sustentam esta pesquisa, abordando os principais conceitos, teorias e estudos relacionados ao tema.

\subsection{Conceito de Crédito}
Crédito, derivado do latim \textit{credere} ou \textit{creditum} (confiança), significa acreditar, confiar e crer. Segundo \cite{rossato2020}, no contexto empresarial, representa a capacidade de pessoas e empresas adquirirem produtos ou serviços com pagamento futuro. \cite{Cardoso2024} complementam que o crédito também pode ser entendido como o montante disponibilizado ao cliente, seja em forma de empréstimo ou financiamento, mediante a promessa de pagamento em data futura.

\cite{ALEXANDRE2003} ressalta que, embora existam diversas definições para o termo crédito ou operação de crédito, é essencial compreender sua origem e sentido etimológico para uma melhor aplicação. Para Luiz Carlos Jacob Perera (apud \cite{ALEXANDRE2003}, "a história do crédito demonstra que sua evolução acompanhou o próprio desenvolvimento econômico da sociedade, procurando desenvolver instrumentos necessários para a satisfação das necessidades e anseios da humanidade. Além disso, o crédito, usado adequadamente, tanto por governos quanto por empresas, como forma de gestão do consumo, continua a mostrar vigor notável, graças ao papel importante que desempenha no cotidiano da humanidade como instrumento provocador e facilitador das transações de bens e serviços."
\subsection{Concessão de Crédito}

A concessão de crédito é sempre uma decisão incerta, porém tem sido um componente importante no desenvolvimento e crescimento da economia e do país. Ela se dá no momento em que a instituição se sente segura ao ponto de entregar a sua mercadoria ou capital, afirma \cite{beserra2022}. 

Para \cite{rossato2020}, é um instrumento para alavancar vendas, disponibilizado por agências financiadoras, cooperativas de crédito e empreendimentos que permitem aos clientes adquirir bens e serviços com pagamento futuro. Além disso, o crédito é destacado como um fator essencial para o desenvolvimento econômico, ao viabilizar a produção de bens e serviços pelos empresários. 

“Fatores como o aumento do grau de estabilidade econômica, surgimento de novos produtos e serviços e controle da inflação contribuem para a ampliação do mercado consumidor”, diz Rodrigo Ventura (apud \cite{fuhr2022}. \cite{fuhr2022} ressaltam que o crescimento de pessoas e empresas no mercado nacional impulsiona e reposiciona a importância da análise de crédito, consolidando seu valor na gestão empresarial e na economia. 

Isso ocorre porque as empresas frequentemente optam por comercializar seus produtos e serviços a prazo, exigindo critérios claros para avaliar e decidir sobre a concessão de crédito. Esse processo envolve a análise do risco de inadimplência, ou seja, a possibilidade de não pagamento dos valores acordados. Segundo Gouvêa Gonçalves (apud \cite{fuhr2022}), “A avaliação do risco tem por objetivo melhorar a qualidade de uma carteira de clientes, favorecendo uma venda saudável e evitando ao máximo a perda de valores, sobre créditos fornecidos de forma equivocada ou a clientes que geram prejuízos aos negócios. Empresas que possuem boa avaliação levam vantagens sobre seus concorrentes”. 

\cite{montevechi2022} citam que a concessão de crédito, fundamental para a indústria financeira, envolve uma relação contratual entre credores e tomadores, assumindo riscos inerentes, como a inadimplência. Para mitigar esses riscos, instituições financeiras desenvolvem metodologias para avaliar a probabilidade de não pagamento e classificar os tomadores com base em dados históricos. Entre essas ferramentas, destaca-se o \textit{credit scoring} (CS), que utiliza modelos matemáticos para prever o comportamento dos devedores e estimar a \textit{probability of default} (PD), contribuindo para a sustentabilidade do sistema creditício e o sucesso das operações financeiras.
\subsection{Avaliação de Risco e Credit Scoring}

Para Gouvêa Gonçalves (apud \cite{fuhr2022}), a avaliação de risco visa melhorar a qualidade da carteira de clientes, evitando perdas por créditos mal concedidos e promovendo vendas saudáveis, conferindo vantagem competitiva às empresas que realizam boas análises. A tecnologia de \textit{credit score} contribuiu significativamente para a diminuição dos custos relacionados à análise de crédito, garantindo consistência nas informações, maior velocidade e acurácia na tomada de decisão.

Os modelos de \textit{credit score} são baseados em informações fornecidas pelos próprios solicitantes e, por meio de técnicas estatísticas e matemáticas, atribuem pontuações que distinguem bons e maus pagadores, segundo \cite{beserra2022}. De acordo \cite{francisco2012}, a classificação do risco de crédito é vista como uma ferramenta fundamental para analistas e instituições.

\cite{fuhr2022} destaca que a classificação de crédito tem como objetivo desenvolver um modelo que integre informações quantitativas e qualitativas sobre a credibilidade da empresa, refletindo a qualidade do devedor. O \textit{credit scoring}, por sua vez, tem como principal propósito avaliar o risco de inadimplência com base em uma pontuação, indicando a probabilidade de um candidato não honrar com seus compromissos acordados.

\subsection{Detecção de Fraudes/Golpes Financeiros}

A fraude financeira representa uma ameaça crescente, com impactos negativos significativos no setor financeiro e na sociedade. Martins e \cite{martins2022} destacam que, embora a mineração de dados seja eficaz na detecção de fraudes, ela enfrenta desafios devido à constante mudança nos perfis comportamentais e à similaridade entre transações fraudulentas e legítimas. Os autores também salientam que a prevenção de fraudes é uma abordagem proativa, focada em evitar sua ocorrência, enquanto a detecção atua de forma reativa, identificando transações fraudulentas em andamento.

Para Santos (apud \cite{soares2024}), "fraude é um processo sistemático de ações cujo objetivo é distorcer dados intencionalmente e que se estabelece quando agentes internos e externos da companhia possuem a intenção de agir dissimuladamente." Conforme \cite{soares2024}, a Associação dos Investigadores de Fraude Certificados (ACFE) classifica as fraudes em três categorias principais:
\begin{enumerate}
    \item \textbf{Corrupção}: Envolve subornos ou uso indevido de bens públicos.
    \item \textbf{Apropriação indevida de ativos}: Inclui fraudes que impactam ou não diretamente o caixa da empresa.
    \item \textbf{Demonstrações financeiras fraudulentas}: Compreendem receitas fictícias e ocultação de passivos e despesas.
\end{enumerate}

Entre as fraudes mais comuns no Brasil, as relacionadas a contas a pagar e contas a receber representam 23\% do total de investigações de fraudes conduzidas, sendo detectadas, em sua maioria, por denúncias anônimas, denúncias nominais e atividades de auditoria interna.

Segundo \cite{thinkdata2024}, a prevenção de fraudes na concessão de crédito é um tema de crescente importância no contexto financeiro brasileiro, especialmente diante do aumento expressivo de tentativas fraudulentas, como roubo de identidade e apresentação de informações falsas. O cenário atual exige que instituições financeiras e empresas adotem estratégias robustas e proativas para identificar e mitigar esses riscos, protegendo a integridade do sistema financeiro e os consumidores. Dados recentes mostram que, em 2022, o Brasil registrou quase 3,9 milhões de tentativas de fraude de identidade, evidenciando a vulnerabilidade do setor financeiro.

Entre os principais desafios da prevenção de fraudes na concessão de crédito estão:
\begin{itemize}
    \item \textbf{Risco de inadimplência}: Requer uma análise precisa da capacidade do cliente de cumprir suas obrigações financeiras, utilizando históricos de crédito e modelos de risco.
    \item \textbf{Avanço tecnológico}: Acompanhamento da evolução das tecnologias para garantir segurança e inovação nos processos de análise.
    \item \textbf{Fraudes sofisticadas}: Enfrentar técnicas cada vez mais avançadas de roubo de identidade e falsificação documental, demandando medidas rigorosas de validação.
\end{itemize}

Práticas como a checagem de dados com operadoras de telefonia, validação de documentos e uso de inteligência artificial oferecem às instituições maior controle e capacidade de resposta aos riscos emergentes.

Conforme mencionado por \cite{maniraj2019}, a fraude, definida como um ato ilícito ou criminal para obtenção de benefícios financeiros ou pessoais, é um desafio significativo no setor financeiro. Diversos estudos exploram técnicas para detecção de fraudes, como mineração de dados, aprendizado supervisionado e não supervisionado, e detecção adversarial. Embora esses métodos tenham alcançado sucesso em algumas áreas, ainda enfrentam limitações na criação de soluções permanentes e consistentes.
\subsection{Detecção de Anomalias}

Conforme mencionado por \cite{gupta2020}, a detecção de anomalias é um método para identificar ocorrências suspeitas de eventos e itens de dados que podem causar problemas para as autoridades competentes. As anomalias nos dados geralmente estão associadas a questões como problemas de segurança, falhas em servidores, fraudes bancárias, falhas estruturais em edifícios, defeitos clínicos, entre outros.

\subsection{Aprendizado de Máquina e Detecção de Fraudes}

De acordo com Maxwell (apud \cite{martins2022}, o aprendizado de máquina (ML – \textit{Machine Learning}) é o estudo de algoritmos de computador que se aprimoram automaticamente com a experiência. É tratado como uma subárea da inteligência artificial (IA). Algoritmos de aprendizado de máquina constroem um modelo baseado em dados de amostra, conhecidos como “dados de treinamento”, a fim de fazer previsões ou decisões sem serem explicitamente programados para isso. Esses algoritmos são usados em uma ampla variedade de aplicações, como filtragem de e-mails e visão computacional, em que é difícil ou inviável desenvolver algoritmos convencionais para realizar as tarefas necessárias.

\cite{martins2022} citam que as principais instituições financeiras, tanto nacionais quanto internacionais, têm adotado algoritmos de aprendizado de máquina para analisar dados, alcançando resultados financeiros significativos. Essa tecnologia tem sido aplicada com sucesso em áreas como análise de risco de crédito, previsão de falências, estimativas de cotações de moedas e ações, segmentação de mercado e detecção de fraudes.

Para Xuan (apud \cite{martins2022}), a detecção de fraudes busca identificar transações atípicas, que podem ocorrer em diferentes contextos, como operações financeiras, consumo de energia, compras, uso de recursos sociais, acesso a redes de computadores e manipulação contábil. Algoritmos de aprendizado de máquina são amplamente utilizados para classificar transações como legítimas ou fraudulentas, configurando um problema de classificação binária. No entanto, a análise enfrenta desafios, como a predominância de transações legítimas nos dados e a constante criação de novas fraudes, exigindo a adaptação contínua dos modelos. Por isso, a detecção de fraudes também é vista como um problema de fluxo contínuo de dados.

\cite{gupta2020} mencionam que o aprendizado de máquina oferece métodos eficazes para extrair informações úteis de grandes volumes de dados, auxiliando na tomada de decisão e aumentando a precisão preditiva. No contexto da detecção de fraudes com cartões de crédito, o foco principal é diferenciar transações fraudulentas das legítimas. O treinamento de sistemas de detecção de fraudes pode ser realizado de três formas:
\begin{enumerate}
    \item \textbf{Supervisionado:} Utiliza conjuntos de dados rotulados, nos quais os itens possuem informações detalhadas e estão previamente classificados. O modelo é treinado para analisar novos dados e realizar a classificação.
    \item \textbf{Semissupervisionado:} Combina dados rotulados e não rotulados, sendo mais utilizado que o método supervisionado. Essa abordagem é ideal para cenários em que há maior disponibilidade de dados não rotulados.
    \item \textbf{Não supervisionado:} Baseia-se em dados não rotulados e identifica padrões anômalos de forma autônoma, assumindo que exceções são raras no conjunto de dados. É a estratégia mais utilizada, especialmente para tarefas mais complexas, embora possa ser mais imprevisível.
\end{enumerate}

Para \cite{bhati2024}, o aprendizado de máquina (ML), um ramo da inteligência artificial (IA), apresenta-se como uma abordagem promissora para a detecção de fraudes em cartões de crédito, permitindo que sistemas aprendam com dados históricos e melhorem suas previsões sem a necessidade de programação manual. No entanto, sua aplicação enfrenta desafios consideráveis, como a anonimização dos dados transacionais, que dificulta a reprodução de estudos, e a natureza dinâmica e desbalanceada das fraudes, que compromete a precisão dos modelos atuais. Esses obstáculos reforçam a necessidade de desenvolver soluções mais eficazes e robustas.


% REFERÊNCIAS
\newpage
\bibliographystyle{abnt-alf}
\bibliography{biblproj} % Certifique-se de que o arquivo biblproj.bib está correto






\end{document}
