\documentclass[12pt,a4paper]{article}
\usepackage[utf8]{inputenc}
\usepackage[brazil]{babel}
\usepackage{graphicx}
\usepackage{hyperref}
\usepackage{abnt-alf}
\usepackage[top=3cm,bottom=2cm,left=3cm,right=2cm]{geometry}
\usepackage{indentfirst}

\begin{document}

% CAPA
\pagestyle{empty}
\begin{center}
\large  \textbf{UNIVERSIDADE PRESBITERIANA MACKENZIE}
\large  \textbf{PROGRAMA DE PÓS-GRADUAÇÃO EM}\\
\large  \textbf{ENGENHARIA ELÉTRICA E COMPUTAÇÃO}\\
\vskip 2.0cm
\textbf{\large João Marcelo Cattaldo Amorim}\\
\vskip 4.0cm
\setlength{\baselineskip}{1.5\baselineskip}
\textbf{\large IDENTIFICAÇÃO DE GOLPES FINANCEIROS EM INDÚSTRIAS DE BENS DE CONSUMO NA CONCESSÃO DE CRÉDITO A PESSOAS  JURÍDICAS COM USO DE INTELIGÊNCIA ARTIFICIAL.}\\
\vskip 4.5cm
\end{center}
\hfill{\vbox{\hsize=8.5cm\noindent\strut
Projeto de Pesquisa apresentado ao Programa\break
de Pós-Graduação em Engenharia Elétrica e\break
Computação da Universidade Presbiteriana\break
Mackenzie como parte dos requisitos para a\break
aprovação na disciplina de Metodologia do\break
Trabalho Científico.}\\
\strut}
\vskip 3.0cm
\textbf{\normalsize Orientador: Prof. Dr. Leandro Augusto da Silva }\\
\vskip 2.0cm
\begin{center}
São Paulo\\
2024\\
\end{center}

% RESUMO
\newpage
\thispagestyle{plain}
\pagenumbering{roman}
\begin{center}
\large
\textbf{RESUMO}
\end{center}
\renewcommand{\baselinestretch}{0.6666666}
Esta dissertação investiga a identificação de padrões associados a golpes financeiros em indústrias de bens de consumo que concedem crédito a clientes pessoa jurídica para a aquisição de seus produtos, utilizando técnicas de inteligência artificial. A análise da base de dados revelou que aproximadamente 15\% dos clientes, cerca de 148.000 empresas, encontram-se em estado de inadimplência não recuperada, representando mais de $R$ 2,1 bilhões em dívidas. Esses clientes possuem um score de crédito classificado como de alto risco. O modelo de crédito atualmente empregado pela empresa apresenta limitações importantes, pois desconsidera fatores críticos que podem indicar golpes na concessão de crédito. Enquanto o credit score tradicional baseia-se principalmente no histórico de pagamentos e nas restrições associadas ao cliente, golpes financeiros geralmente envolvem alterações cadastrais recentes, como mudanças de endereço ou código de atividade econômica, além de comportamentos atípicos, incluindo aumentos súbitos no volume de compras e consultas, bem como discrepâncias no tempo de atividade em relação à data de fundação da empresa.
Para abordar esses desafios, esta pesquisa desenvolveu uma solução baseada em modelos de machine learning, incluindo Classification Based on Associations (CBA) e Random Forest para tarefas de classificação, e técnicas avançadas de detecção de anomalias, como Isolation Forest, XGBoost, Autoencoders e One-Class SVM. No entanto, os resultados mostraram que o uso exclusivo de classificadores supervisionados, como o CBA e o Random Forest, não foi satisfatório devido à natureza do problema e às características dos dados, como o elevado desbalanceamento e a limitação das regras de classificação mais utilizadas, que não representaram adequadamente a detecção de fraudes.
Entre as técnicas de detecção de anomalias, o Isolation Forest foi o modelo mais eficaz, validando as principais hipóteses relacionadas aos padrões de fraudes. Essa abordagem demonstrou-se eficiente para identificar comportamentos atípicos e padrões indicativos de golpes financeiros, contribuindo significativamente para o aprimoramento da avaliação de risco de crédito e para a mitigação de perdas financeiras associadas a fraudes. O estudo reforça a importância de combinar técnicas supervisionadas e não supervisionadas para desenvolver soluções robustas e oferece uma contribuição relevante tanto para a literatura acadêmica quanto para a prática empresarial.


\\[0.5cm]
\begin{flushleft}
{\bf Palavras-chave:} {\it Análise de Crédito, Risco de Crédito, Inadimplência, Indústrias de Bens de Consumo, Credit Score, Aprendizado de Máquina, Ciência de Dados, Análise de Dados, CRISP-DM.}
\end{flushleft}

% SUMÁRIO
\newpage
\thispagestyle{empty}
\tableofcontents

% DESENVOLVIMENTO
\newpage
\pagestyle{plain}
\pagenumbering{arabic}
\renewcommand{\baselinestretch}{1.5}
\normalsize
\section{RESUMO}
Esta dissertação explora a identificação de padrões relacionados a golpes financeiros em indústrias de bens de consumo que concedem crédito a clientes pessoa jurídica para a aquisição de seus produtos, utilizando técnicas de inteligência artificial. A análise da base de dados revelou que aproximadamente 15% dos clientes, cerca de 148.000 empresas, encontram-se em estado de inadimplência não recuperada, representando mais de R$ 2,1 bilhões em dívidas. Esses clientes possuem um score de crédito classificado como de alto risco. O modelo de crédito atualmente empregado pela empresa apresenta limitações importantes, pois desconsidera fatores críticos que podem indicar golpes na concessão de crédito. Enquanto o credit score tradicional baseia-se principalmente no histórico de pagamentos e nas restrições associadas ao cliente, golpes financeiros geralmente envolvem alterações cadastrais recentes, como mudanças de endereço ou código de atividade econômica, além de comportamentos atípicos, incluindo aumentos súbitos no volume de compras e consultas, bem como discrepâncias entre o tempo de atividade e a data de fundação da empresa.



\section{INTRODUÇÃO}
As indústrias de bens de consumo concedem crédito para as empresas que querem adquirir seus produtos no mercado nacional. A análise e concessão de crédito referem-se ao processo pelo qual uma instituição avalia a capacidade de um indivíduo, empresa ou entidade de cumprir com obrigações financeiras e decide se concederá crédito a essa parte.

Durante a análise de crédito, diversos fatores são considerados, tais como histórico de crédito, porte, histórico de pagamentos e outras informações financeiras relevantes. Com base nesses dados, a instituição determina o nível de risco associado à concessão de crédito e decide se aprova ou não a solicitação de crédito.

A concessão de crédito é o processo pelo qual a instituição financeira ou credora decide fornecer um empréstimo, uma linha de crédito ou outra forma de financiamento ao requerente, com base nos resultados da análise de crédito. Este processo pode variar dependendo da política de crédito da instituição e das condições econômicas gerais.

O Credit Score, ou pontuação de crédito, é um número que representa a avaliação do risco de crédito de um consumidor ou empresa. Esse número é calculado com base em informações financeiras disponíveis, como histórico de pagamento de dívidas, montante de dívidas, tempo de crédito, tipos de crédito utilizados e outras variáveis relevantes. É frequentemente utilizado por instituições financeiras e credores para determinar a probabilidade de um indivíduo ou empresa pagar suas dívidas pontualmente. 

A pesquisa é motivada pela necessidade de aprimorar o modelo de Credit Score atualmente em uso na empresa, desenvolvido em meados de 2000 e ainda amplamente empregado. Com o surgimento de novas fontes de informação e a explosão na geração de dados, torna-se imperativo identificar as principais variáveis e seus respectivos pesos para aprimorar as decisões de crédito. 

Minha experiência profissional no desenvolvimento de sistemas e algoritmos desempenha um papel fundamental na minha motivação para impulsionar os produtos e serviços da empresa. Essa bagagem permite-me contribuir significativamente para o aprimoramento e inovação de nossos produtos, garantindo que estejamos sempre à frente das demandas do mercado e das necessidades dos clientes.
\section{PROBLEMA DE PESQUISA}
As indústrias de bens de consumo afiliadas à CISP, uma organização sem fins lucrativos composta por 192 grandes empresas, são reconhecidas por sua liderança no desenvolvimento de soluções técnicas e tecnológicas para a gestão de risco de crédito. O faturamento combinado dessas indústrias representa 8 por cento do PIB nacional, com um total de 1.400.000 CNPJs cadastrados. A análise de crédito realizada por essas indústrias, direcionada aos clientes em busca de adquirir seus produtos, tem como principal critério o Credit Score.

O Credit Score, desenvolvido em 2000, permanece amplamente utilizado até os dias de hoje. Com o crescente volume de dados gerados, surgimento de novos mercados e modelos de negócios, assim como diversas fontes adicionais de informação, é essencial compreender quais variáveis e seus respectivos pesos são cruciais para a determinação de um Credit Score eficaz atualmente.
\section{CENÁRIO EXEMPLO}
Para contextualizar a aplicação prática do projeto de pesquisa sobre o aprimoramento do modelo de Credit Score para as indústrias de bens de consumo associadas à CISP, consideremos uma situação hipotética na indústria de alimentos.

Imagine uma grande fabricante nacional de alimentos, denominada ficticiamente Alimentos S/A, que é uma das associadas à CISP. A  Alimentos S/A possui uma ampla gama de produtos, desde arroz, feijão e até carne bovina e cortes de frango. Como uma das líderes de mercado, a empresa atrai uma extensa base de clientes em todo o país.
No entanto, apesar da sua posição consolidada no mercado, a Alimentos S/A enfrenta desafios significativos relacionados à concessão de crédito aos seus clientes. Muitas empresas buscam crédito para adquirir os produtos da empresa, o que aumenta a necessidade de avaliação precisa do risco de crédito.

Atualmente, a Alimentos S/A utiliza o modelo de Credit Score fornecido pela CISP para avaliar a capacidade de pagamento de seus clientes. No entanto, o modelo atual mostra algumas limitações na identificação de padrões nas informaçoes restririvas e comportamento de pagamento específicos do setor de alimentos, como sazonalidade de compras, influência de promoções sazonais e impacto de fatores econômicos regionais.

Como resultado, a Alimentos S/A está interessada em colaborar com a pesquisa proposta para aprimorar o modelo de Credit Score. Eles acreditam que um modelo mais preciso e personalizado pode não apenas reduzir o risco de inadimplência, mas também otimizar a oferta de crédito, aumentando assim as vendas e fortalecendo sua posição competitiva no mercado.

Este cenário exemplifica a relevância prática da pesquisa e destaca a importância de adaptar o modelo de Credit Score às especificidades de cada setor alimentício, visando melhorar a eficácia das decisões de crédito e impulsionar o desempenho financeiro das empresas.
\section{JUSTIFICATIVA}
Devido ao aumento da inadimplência entre as empresas no mercado nacional, as indústrias que concedem crédito precisam realizar uma gestão mais eficaz do risco de crédito de sua carteira de clientes. O aprimoramento do Credit Score traz mais segurança e assertividade na tomada de decisão, ajudando a identificar potenciais riscos e proteger os interesses da empresa contra possíveis perdas financeiras.

Em linhas gerais a cada cliente inadimplente, a empresa precisa realizar cerca de 10 bons negócios para reduzir suas perdas.
\section{CONCEITOS FUNDAMENTAIS}
A análise de crédito é o processo pelo qual as instituições financeiras e outras entidades avaliam a capacidade de crédito de um cliente potencial antes de conceder um empréstimo, financiamento ou linha de crédito. Durante esse processo, são considerados diversos fatores, como histórico de pagamento, renda, histórico de crédito, dívidas pendentes e outros, a fim de determinar a probabilidade de o cliente honrar suas obrigações financeiras.

O risco de crédito refere-se à possibilidade de uma instituição financeira ou credor sofrer perdas financeiras devido à inadimplência de um devedor. Em outras palavras, é o risco de o cliente não pagar o empréstimo ou financiamento conforme acordado. O risco de crédito é avaliado durante a análise de crédito e é influenciado por diversos fatores, como o histórico de pagamento do cliente, sua capacidade de pagamento, o valor do empréstimo e a condição econômica geral.

O Credit Score, ou pontuação de crédito, é uma medida numérica que avalia a probabilidade de um indivíduo ou empresa honrar suas obrigações financeiras com base em seu histórico de crédito e outros fatores relevantes. Essa pontuação é calculada por agências de crédito com base em informações como histórico de pagamento, dívidas pendentes, histórico de crédito, tempo de crédito, tipos de crédito utilizados e novas solicitações de crédito. O Credit Score é utilizado por instituições financeiras para tomar decisões de crédito, determinando os termos e condições do empréstimo, como taxa de juros, prazo de pagamento e limite de crédito. Uma pontuação mais alta indica um menor risco de inadimplência, enquanto uma pontuação mais baixa indica um maior risco.

O CRISP-DM (Cross-Industry Standard Process for Data Mining) é um modelo padrão para projetos de mineração de dados que fornece uma estrutura abrangente e passo a passo para guiar os profissionais de dados em todas as etapas do processo. O CRISP-DM é composto por seis fases principais: Entendimento do Negócio, Entendimento dos Dados, Preparação dos Dados, Modelagem, Avaliação e Implantação.

Aprendizado de máquina, por outro lado, é uma subárea da inteligência artificial que se concentra no desenvolvimento de algoritmos e técnicas que permitem aos computadores aprenderem a partir de dados e fazerem previsões ou tomar decisões sem serem explicitamente programados para isso. O aprendizado de máquina utiliza métodos estatísticos e computacionais para identificar padrões nos dados e construir modelos que possam ser usados para fazer previsões ou tomar decisões em novos dados.

Portanto, o CRISP-DM fornece uma estrutura para conduzir projetos de mineração de dados, enquanto o aprendizado de máquina é uma técnica frequentemente utilizada dentro desse contexto para desenvolver modelos preditivos ou descritivos a partir dos dados disponíveis. O CRISP-DM e o aprendizado de máquina são frequentemente combinados em projetos de análise de dados para resolver problemas de negócio complexos e extrair insights valiosos dos dados.
\section{OBJETIVO DA PESQUISA}
Aperfeiçoamento do modelo atual de Credit Score direcionado às indústrias de bens de consumo associadas à CISP, utilizando técnicas avançadas de aprendizado de máquina. O objetivo é avaliar o risco de crédito de empresas no mercado nacional
A fundamentação teórica estabelece os contornos da problemática da pesquisa por meio da descrição/definição de conceitos essenciais ao tema estudado, da apresentação de um panorama histórico ou mesmo de uma compilação de aspectos tratados em outras pesquisas e do delineamento das lacunas encontradas na literatura.

A bibliografia selecionada para a construção do quadro teórico deverá considerar a relevância e atualidade em relação ao tema em questão. Ao longo da apresentação da revisão da literatura deve-se demonstrar o entendimento e articulação entre os conceitos e definições que fazem parte da área e da especificidade da pesquisa que será desenvolvida. Neste sentido, esta parte do projeto também comporta, conforme o caso, subitens temáticos que possibilitem a melhor compreensão do contexto da investigação.

De maneira complementar, a apresentação de conceitos, definições, etc., deve ser feita por meio de paráfrase, sendo a referência utilizada incluída no formato ``(AUTOR, Ano)''. A não apresentação da referência ou cópia literal de elementos textuais sem a indicação da fonte bibliográfica pertinente constitui plágio e é passível de punições, inclusive de reprovação na disciplina.

Constam da {\it bibliografia básica} os elementos bibliográficos referenciados ao longo do projeto de pesquisa e outras bibliografias que ainda serão consultadas e estudas e ao longo do desenvolvimento do trabalho. A formatação da bibliografia básica deverá obedecer às regras da ABNT (NBR 6023/2015) e ser disposta em ordem alfabética (conforme exemplo apresentado no final deste documento
\section{PROPOSTA DA SOLUÇÃO}
Devido ao aumento da inadimplência entre as empresas no brasil, as indústrias de bens de consumo que concedem crédito para seus clientes adquirem seus produtos, precisam realizar uma gestão mais eficaz do risco de crédito de sua carteira de clientes. O aprimoramento do Credit Score traz mais segurança e assertividade na tomada de decisão, ajudando a identificar potenciais riscos e proteger os interesses da empresa contra possíveis perdas financeiras.

\section{METODOLOGIA}
Utilizarei uma abordagem exploratória devido à quantidade de informações disponíveis. Será realizada uma análise minuciosa em relação aos tipos de variáveis que podem ser utilizadas e que façam sentido para a análise de risco.

Os dados serão coletados em formato CSV para facilitar o manuseio, com a escolha de um grupo de associados para conduzirmos o estudo. Esses dados incluirão campos para identificação dos clientes e outros campos descrevendo o comportamento de pagamentos. A base de dados para o estudo deverá conter clientes em comum entre os associados escolhidos, visando apurar com mais precisão e generalizar o modelo.

A população do estudo consistirá em aproximadamente 200.000 empresas de vários segmentos e portes distintos, com o intuito de diversificar o modelo. A amostra será composta por empresas que realizaram pelo menos uma compra nos últimos 12 meses e que estão ativas na Receita Federal.

A coleta dos dados será realizada através da extração direta dos sistemas ERP dos associados. Para extrair o arquivo em formato CSV, cada ERP possui a ferramenta mais adequada. Caso não esteja disponível, a realização de uma instrução SQL será eficiente para obter os dados.

Os dados serão analisados com estatística descritiva e por meio da plotagem de gráficos de dispersão, histogramas e boxplot para entender e validar a amostra. Após a análise descritiva dos dados, será realizada uma análise de identificação de correlação entre as variáveis e, posteriormente, a fase de engenharia de características nos trará resultados significativos para a escolha das variáveis do modelo.

As informações coletadas serão anonimizadas por questões éticas e de privacidade, a fim de não identificar quais empresas e clientes estão sendo categorizados.
\section{CONCLUSÃO}
Devido ao aumento da inadimplência entre as empresas no brasil, as indústrias de bens de consumo que concedem crédito para seus clientes adquirem seus produtos, precisam realizar uma gestão mais eficaz do risco de crédito de sua carteira de clientes. O aprimoramento do Credit Score traz mais segurança e assertividade na tomada de decisão, ajudando a identificar potenciais riscos e proteger os interesses da empresa contra possíveis perdas financeiras.
\section{CRONOGRAMA}
\begin{enumerate}
    \item Entendimento do Negócio (Maio/24 até Maio/24)
    \item Entendimento dos Dados (Junho/24 até Julho/24)
    \item Preparação dos Dados 	(Julho/24 até Agosto/24)
    \item Modelagem	(Setembro/24 até Setembro/24)
    \item Avaliação	(Outubro/24 até Outubro/24)
    \item Implantação (Novembro/24 até Fevereiro/25)
\end{enumerate}
\def\refname{REFERÊNCIAS BIBLIOGRÁFICAS}
\bibliography{biblproj}
\addcontentsline{toc}{section}{REFERÊNCIAS BIBLIOGRÁFICAS}
\bibliographystyle{abnt-alf}
\end{document}
